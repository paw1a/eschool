\chapter{Аналитическая часть}

В данной части проводится анализ существующих решений, формализация данных, ролей и задачи, анализ баз данных.

\section{Анализ существующих решений}
На рынке онлайн образования представлено множество платформ, предоставляющих образовательные материалы разной тематики и формы. Ключевым криетрием, по которому были отобраны существующие решения, является наличие на платформе как теории, так и практики. При анализе отобранных решений были выделены следующие критерии:
\begin{itemize}
    \item наличие подробного отчета о пройденном обучении;
    \item возможность создать и опубликовать на платформе собственный курс;
    \item понятный и простой пользовательский интерфейс;
    \item возможность основать собственную школу.
\end{itemize}

\begin{table}[H]
    \caption{\label{tbl:alternatives}Сравнение существующих решений}
	\resizebox{\textwidth}{!}{
	\def\arraystretch{1}
    \begin{tabular}{|c|c|c|c|c|}
    \hline
    Решение          & Подробный отчет & Создание курса & Собственная школа & Простой польз. интерфейс \\ \hline
    Stepik           & +               & +              & -                 & +                        \\ \hline
    Coursera         & -               & -              & -                 & -                        \\ \hline
    Яндекс Практикум & +               & -              & -                 & +                        \\ \hline
    \end{tabular}%
    }
\end{table}

Из таблицы \ref{tbl:alternatives} видно, что существующие решения не удовлетворяют описанным требованиям в полной мере.

\section{Формализация данных}
В предметной области онлайн образования ключевыми сущностями являются школа и курс. Школа в данной работе включает в себя множество преподавателей, студентов и курсов, которые могут быть реализованы
преподавателями данной школы. Также школа содержит дополнительную информацию о создателе, платежные данные
для оплаты того или иного курса и прочую дополнительную информацию.

Каждый курс разбит на уроки, которые могут быть нескольких типов: текстовые, видео-уроки и практические.
Текстовые и видео уроки предоставляют теоретические знания, в то время как практические задания
реализуются в виде тестов с множественным выбором.

База данных должна хранить информацию о следующих сущностях:
\begin{itemize}
    \item пользователи;
    \item школы;
    \item курсы;
    \item уроки, принадлежащие некоторому курсу;
    \item практические тесты;
    \item отзывы;
    \item сертификаты.
\end{itemize}

На рисунке \ref{img:er} приведена ER-диаграмма сущностей в нотации Чена.

\begin{figure}[H]
	\centering
	\includegraphics[height=0.9\textheight]{inc/img/er.pdf}
	\caption{ER-диаграмма сущностей в нотации Чена}
	\label{img:er}
\end{figure}

\section{Формализация ролей}
В данной работе выделяются следующие роли пользователей:
\begin{itemize}
    \item гость -- неавторизованный пользователь, который может посмотривать информацию о курсе,
    авторизоваться, зарегистрироваться;
    \item пользователь -- авторизованный пользователь, может просматривать курсы, проходить, оплачивать их,
    по завершении курса получать сертификат и отчет об успеваемости и оставлять отзывы,
    стать преподавателем, создать собственную школу и курсы в ней,
    а также присоединиться к существующей школе в качестве преподавателя;
    \item администратор -- авторизованный пользователь, может создавать, изменять и удалять пользователей, курсы, школы, уроки, отзывы.
\end{itemize}

\section{Формализация задачи}
Необходимо спроектировать базу данных для хранения информации о студентах и преподавателях, школах, курсах и уроках, входящих в состав курса, а также об отзывах пользователей.
Разработанное приложение для доступа к базе данных должно предоставлять каждому пользователю возможность покупать и проходить курсы различной тематики. Также у каждого пользователя должна быть возможность создать собственную школу и образовательную программу
или присоединиться к уже существующей школе в качестве преподавателя или составителя курса.

По прохождении курса пользователь должен иметь возможность получить сертификат о завершении обучения,
включающий подробное описание его успеваемости и итоговую оценку. Диаграмма вариантов использования приведена на рисунке \ref{img:usecase}

\begin{figure}[H]
	\centering
	\includegraphics[height=0.7\textheight]{inc/img/usecase.pdf}
	\caption{Диаграмма вариантов использования}
	\label{img:usecase}
\end{figure}

\section{Анализ баз данных}
База данных -- совокупность данных, хранимых в соответствии со схемой данных, манипулирование которыми выполняют в соответствии с правилами средств моделирования данных.

СУБД -- совокупность языковых и программных средств общего или специального назначения, обеспечивающих управление созданием и использованием баз данных~\cite{williams}.

Модель данных -- это абстрактное, самодостаточное, логическое определение объектов, операторов и прочих элементов, в совокупности составляющих абстрактную машину доступа к данным, с которой взаимодействует пользователь. Эти объекты позволяют моделировать структуру данных, а операторы -- поведение данных~\cite{williams}.

По модели хранения базы данных делятся на три группы.
\begin{enumerate}
    \item Дореляционные модели. Основные представители:
    \begin{itemize}
        \item иерархические;
        \item сетевые;
        \item инвертированные списки.
    \end{itemize}
    \item Реляционные модели данных.
    \item Постреляционные модели данных.
\end{enumerate}

\subsection*{Дореляционные модели}
Дореляционные базы данных представляют собой тип баз данных, который не использует табличную модель данных. Вместо этого, они хранят данные в структурах, таких как деревья, графы или объекты. 

\subsection*{Реляционная модель данных}
Реляционные базы данных основаны на реляционной модели данных, где данные организованы в виде таблиц, которые имеют связи между собой. Каждая таблица представляет отдельную сущность, а связи между таблицами устанавливаются с помощью ключевых полей. Реляционные базы данных обеспечивают управление данными, обеспечивая стандартизированный способ доступа и операций с данными. Важным свойством реляционных баз данных является их способность удовлетворять требованиям ACID~\cite{acid}: атомарность, согласованность, изоляция, устойчивость. Примерами реляционных баз данных являются MySQL, PostgreSQL, Oracle и SQL Server.

\subsection*{Постреляционная модель данных}
Постреляционные базы данных -- это базы данных, которые разработаны для обработки и анализа больших объемов данных. Они предлагают расширенные возможности для работы с неструктурированными данными, имеют гибкую схему данных. Примерами постреляционных баз данных являются Apache Hadoop, Apache Cassandra и Apache Spark.

\section{Вывод из аналитической части}
С учетом задачи была выбрана реляционная модель хранения данных, так как предметная область может быть представлена в виде таблиц и должна удовлетворять требованиям ACID.
