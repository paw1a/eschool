\chapter*{Заключение}
\addcontentsline{toc}{chapter}{ЗАКЛЮЧЕНИЕ}

В ходе выполнения данной курсовой работы были выполнены следующие задачи:
\begin{itemize}
    \item описана структура трехмерной сцены и определен формат задания исходных данных;
    \item выбран и модифицирован наиболее подходящий из существующих алгоритмов трехмерной графики, позволяющих синтезировать изображение трехмерной сцены;
    \item реализованы выбранные алгоритмы построения трехмерной сцены;
    \item исследованы возможности микроконтроллеров в области визуализации трехмерной графики.
\end{itemize}

Однако в процессе разработки макета устройства были обнаружены некоторые проблемы, связанные с питанием. Для работы шести дисплеев требуется большая мощность источника питания, по сравнению с той, которая предоставляется микроконтроллером (3.3 В). Для работы макета устройства необходимо использовать стационарный источник питания с напряжением 12 В. В связи с отсутствием такого источника, аппаратный комплекс был переработан и количество дисплеев сокращено до двух. С учетом этого факта, архитектура устройства была переработана. В результате макет устройства включает в себя микроконтроллер и два экрана, задающих две соседние грани куба, на каждой из которых изображается проекция трехмерной сцены.   

В качестве дальнейшего развития можно рассмотреть доработку макета с учетом обнаруженных проблем. Таким образом, при наличии подходящего источника питания, устройство будет работать с шестью экранами. Разработанная кодовая база позволяет дальнейшее развитие программно-аппаратного комплекса.
