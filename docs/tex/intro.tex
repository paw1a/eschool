\chapter*{Введение}
\addcontentsline{toc}{chapter}{ВВЕДЕНИЕ}

В современном мире электронные образовательные платформы становятся все более популярными и широко используются для обучения и самосовершенствования. Автоматизация анализа информации об обучающихся, курсах, материалах и результатов обучения на таких платформах играет ключевую роль в обеспечении эффективного функционирования системы. В связи с увеличением количества создаваемых обучающих материалов, актуальной задачей становится разработка системы управления образовательной деятельностью, а также обеспечение безопасности и целостности данных.

Цель данной работы -- разработка базы данных электронной образовательной платформы.

Чтобы достичь поставленной цели, требуется решить следующие задачи: 
\begin{itemize}
    \item проанализировать существующие решения;
    \item спроектировать базу данных;
    \item спроектировать приложение доступа к базе данных;
    \item реализовать приложение доступа к базе данных;
    \item исследовать.
\end{itemize}
