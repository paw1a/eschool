\chapter{Конструкторская часть}

В данном разделе будут описаны алгоритмы и структуры данных, выбранные для
решения поставленной задачи, будет разработана структура программно-аппаратного комплекса. 

\section{Структуры данных}

Чтобы формализовать алгоритм синтеза изображения в программе, необходимо определить структуры данных, которые будут в ней использоваться. 
В данной работе приняты следующие соглашения. 
Трёхмерные модели являются полигональными, тогда сцену можно представить в виде массива многоугольников (полигонов). Многоугольник включает в себя следующие данные:

\begin{itemize}
    \item количество вершин;
    \item массив x и y координат вершин;
    \item коэффициенты уравнения поверхности, несущей данный многоугольник заданного в виде $a \cdot x + b \cdot y + c \cdot z = d$;
    \item цвет в цветовой модели RGB~\cite{rgb}.
\end{itemize}

Окна, использующиеся в алгоритме Варнока, имеют прямоугольную форму и хранят следующую информацию:
\begin{itemize}
    \item количество многоугольников, рассматриваемых при изображении данного окна;
    \item массив многоугольников;
    \item координаты x и y левой верхней и правой нижней вершин окна. 
\end{itemize}

Структура камеры содержит:
\begin{itemize}
    \item позицию камеры в объектном пространстве;
    \item координаты точки, в которую направлен обзор камеры;
    \item вектор, направленный вверх;
    \item вектор, направленный вправо.
\end{itemize}

Сцена состоит из объектов в пространстве, камеры и источников света.

\section{Трехмерные преобразования}
Для корректной работы алгоритмов удаления невидимых линий и поверхностей сначала необходимо провести преобразования объектов сцены, заданных в объектном пространстве.

\subsection{Пространство камеры}
Первым преобразованием является переход из системы координат глобального пространоства в пространство камеры (англ. Camera view) \cite{camera}.

\begin{equation}
M_{view} = 
\begin{bmatrix}
    R_x & R_y & R_z & 0 \\
    U_x & U_y & U_z & 0 \\
    D_x & D_y & D_z & 0 \\
    0   &  0  &  0  & 0 \\  
\end{bmatrix}
\times
\begin{bmatrix}
    1 & 0 & 0 & -P_x \\
    0 & 1 & 0 & -P_y \\
    0 & 0 & 1 & -P_z \\
    0 & 0 & 0 & 1    \\  
\end{bmatrix},
\end{equation}
где $U$ -- вектор, задающий камеру и направленный вверх, $R$ -- вектор, задающий камеру и направленный вправо, $D$ -- вектор направления обзора камеры, $P$ -- вектор позиции камеры. $M_{view}$ -- итоговая матрица преобразования произвольного вектора из глобальной системы координат в систему координат камеры.

\subsection{Перспективные преобразования}
Для построения реалистичных изображений необходимо учитывать перспективу трехмерной сцены. Для этого объекты, находящиеся в пространстве камеры, подвергаются перспективному преобразованию.

\begin{equation}
M_{proj} = 
\begin{bmatrix}
    \frac{1}{ar \cdot \tg{\frac{\alpha}{2}}} & 0 & 0 & 0 \\
    0 & \frac{1}{\tg{\frac{\alpha}{2}}} & 0 & 0 \\
    0 & 0 & \frac{-NearZ-FarZ}{NearZ-FarZ} & \frac{2 \cdot FarZ \cdot NearZ}{NearZ-FarZ} \\
    0   &  0  &  1  & 0 \\  
\end{bmatrix},
\end{equation}
где $ar$ -- соотношение сторон изображения (англ. aspect ration), $NearZ$ и $FarZ$ -- координаты Z, ограничивающие область преобразования, $\alpha$ -- угол обзора. $M_{proj}$ -- матрица перспективного преобразования \cite{perspective}. 

Итоговое преобразование вектора из глобальной системы координат, в систему координат, требующуюся для синтеза изображения можно записать в виде формулы~\ref{transform}:
\begin{equation}
V_{result} = M_{proj} \cdot M_{view} \cdot V_{global},
\label{transform}
\end{equation}
где $V_{result}$ -- итоговый вектор, $V_{global}$ -- вектор в глобальной системе координат~\cite{projection}.

\section{Структура аппаратного комплекса}
Макет устройства представляет собой макетную плату, к которой подключены микроконтроллер Raspberry Pi Pico, шесть TFT дисплеев с разрешением 240x240 пикселов, а также кнопка перезапуска микроконтроллера. Периферия подключена к шине SPI с тактовой частотой 40 МГц. Для обеспечения выбора одного из шести дисплеев для отрисовки текущего кадра используется отдельный пин Chip Select~\cite{cs}. Питание платы осуществляется за счет USB провода, подключенного к микроконтроллеру. 

\section{Алгоритм работы программно-аппаратного комплекса}

Алгоритм работы программно-аппаратного комплекса представлен в виде диаграммы, оформленной в
соответствии с нотацией IDEF0 и отражающей общую декомпозицию
алгоритма~\cite{idef0}.

\begin{figure}[H]
	\centering
	\includegraphics[height=0.45\textheight]{inc/img/01_A0.pdf}
	\caption{Функциональная схема программно-аппаратного комплекса, декомпозиция верхнего уровня}
	\label{fig:a01}
\end{figure}

\begin{figure}[H]
	\centering
	\includegraphics[height=0.45\textheight]{inc/img/02_A0.pdf}
	\caption{Функциональная схема программно-аппаратного комплекса, декомпозиция уровня A0}
	\label{fig:a02}
\end{figure}

\begin{figure}[H]
	\centering
	\includegraphics[height=0.45\textheight]{inc/img/03_A1.pdf}
	\caption{Функциональная схема программно-аппаратного комплекса, декомпозиция уровня A1}
	\label{fig:a1}
\end{figure}

\begin{figure}[H]
	\centering
	\includegraphics[height=0.45\textheight]{inc/img/04_A2.pdf}
	\caption{Функциональная схема программно-аппаратного комплекса, декомпозиция уровня A2}
	\label{fig:a2}
\end{figure}

\begin{figure}[H]
	\centering
	\includegraphics[height=0.45\textheight]{inc/img/05_A4.pdf}
	\caption{Функциональная схема программно-аппаратного комплекса, декомпозиция уровня A4}
	\label{fig:a4}
\end{figure}

\section{Алгоритм Варнока}
Для удаления невидимых линий и поверхностей был выбран алгоритм Варнока, который основан на рекурсивном разбиении окон. 
Окно представляет из себя некоторую прямоугольную подобласть итогового изображения.
В данной работе будут рассматриваться две итерационные реализация алгоритма. Классическая реализация алгоритма Варнока -- реализация, при которой начальное окно разделяется на четыре равные части на каждом уровне рекурсии. Такая реализация ползволяет решить поставленную в данной работе задачу, однако такой метод решения не является эффективным по времени исполнения. Будет также рассмотрен модифицированный алгоритм Варнока, использующий возможности аппаратной платформы более эффективно. 

\subsection{Классический алгоритм Варнока}
Данный алгоритм представляет из себя итеративную реализацию, использующую стек для избежания рекурсии. Изображение для каждого из шести экранов на макете устройства синтезируется и отображается на экран последовательно, что приводит к большим задержкам отрисовки. 

Схема алгоритма Варнока представлена на рисунке \ref{fig:warnock}.

\begin{figure}[H]
	\centering
	\includegraphics[height=0.75\textheight]{inc/img/warnock.pdf}
	\caption{Схема работы алгоритма Варнока}
	\label{fig:warnock}
\end{figure}

\subsection{Модифицированный алгоритм Варнока}
Модифицированная реализация алгоритма позволяет использоваться такую технологию, как прямой доступ к памяти (англ. Direct memory access)~\cite{dma}. Поскольку в микроконтроллере малый объем оперативной памяти, то буфер кадра будет разделен на шесть сегментов, каждый из которых отвечает за прямоугольную подобласть экрана. Таким образом изображение будет синтезироваться для каждого экрана по частям, при этом каждая область изображения отправляется на контроллер экрана с использованием прямого доступа к памяти. Преимуществом такого подхода является то, что прямой доступ к памяти не задействует процессорное время, следовательно при отправке части изображения, микроконтроллер может продолжать синтезировать оставшиеся области. 

Схема модифицированного алгоритма Варнока представлена на рисунке~\ref{fig:warnock_modified}.
\begin{figure}[H]
	\centering
	\includegraphics[height=0.75\textheight]{inc/img/warnock_modified.pdf}
	\caption{Схема работы модифицированного алгоритма Варнока}
	\label{fig:warnock_modified}
\end{figure}

\section{Вывод из конструкторской части}
В данном разделе были описаны алгоритмы и структуры данных, выбранные для решения поставленной задачи и отвечающие заданным требованиям. 
